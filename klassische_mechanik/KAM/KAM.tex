%Schriftgr{\"o}{\ss}e, Layout, Papierformat
\documentclass[12pt,a4paper]{article}

\usepackage[latin1]{inputenc}   %Umlaute erm{\"o}glichen
\usepackage[ngerman]{babel} %deutsches Paket

\usepackage{amssymb}%Mathemodus
\usepackage{amsmath}
\usepackage{latexsym}
\usepackage{algorithmic}
\usepackage{textcomp}
\usepackage[T1]{fontenc}
\usepackage{bm}% bold math
\usepackage{hyperref}
\usepackage{amsfonts}
\usepackage{graphics}
\usepackage{psfrag}
\usepackage[bf,small,sf]{caption}
\usepackage{graphicx}
\usepackage{epsfig}
\usepackage{rotating}
\usepackage[latin1]{inputenc}
\usepackage{color}
\usepackage[usenames,dvipsnames]{pstricks}
\usepackage{epsfig}
\usepackage{pst-grad} % For gradients
\usepackage{pst-plot} % For axes
%Seite einrichten
\usepackage[left=2cm,right=3cm,top=2cm,bottom=2cm,includeheadfoot]{geometry}
\parindent0mm % Unterdr{\"u}ckt die Absatzeinr{\"u}ckung
%Kopf- und Fu{\ss}zeilen
\usepackage{fancyhdr}                   %Paket f{\"u}r Kopf- und Fusszeile
\usepackage[section]{placeins}
\pagestyle{fancy} \fancyhf{}
%Kopfzeile
\fancyhead[L]{\small{\nouppercase{\leftmark}}}%
\fancyhead[R]{\small{St�rungstheorie - KAM-Theorem}
%Linie oben
\renewcommand{\headrulewidth}{0.5pt}
%Fu�zeile
\fancyfoot[C]{\small{\thepage}}

%%%%%%%%%%%%%%%%%%%%%%%%%%%%%%%%%%%%%%%%%%%%%%%%%%%%%%%%%
%ein paar Farben definieren
\definecolor{lightblue}{rgb}{0.3,0.85,1}
%Kommentarfunktion
\newcommand{\jnote}[1]{\textcolor{red}{\textit{[J: #1]}}}
\newcommand{\vnote}[1]{\textcolor{blue}{\textit{[V: #1]}}}
\newcommand{\fnote}[1]{\textcolor{lightblue}{\textit{[F: #1]}}}
\newcommand{\tnote}[1]{\textcolor{green}{\textit{[T: #1]}}}
%%%%%%%%%%%%%%%%%%%%%%%%%%%%%%%%%%%%%%%%%%%%%%%%%%%%%%%%%

%Dokumentbeginn
\begin{document}

%Titelseite
\begin{titlepage}
\begin{center}
\begin{large}
\begin{tabular}{ll}\\\\\\
\multicolumn{2}{p{14.0cm}}{\centering \huge\bf St�rungstheorie - KAM-Theorem
}\\[20mm]
\hline \\
\textbf{Name}:    & Volker Karle\\[6mm]
\textbf{Datum:}   & \today\\[5mm] %oder \today

\hline \\[35mm]

\end{tabular}

\end{large}
\end{center}
\end{titlepage}

\newpage
%Inhaltsverzeichnis
\tableofcontents
\newpage
\setcounter{page}{1}
\part{Wahloser Text}
\section{Kurzfassung}
Beispielzitat: \cite{Beispielbuch} 
Beispielformel in Text $1+1=2$
\begin{figure}[!h]\centering

    \includegraphics[width=12cm]{Beispielbild.jpg}

   \caption{
	   blablabla
}
\label{fig1:Beispieltext}
\end{figure}
\subsection{Unterkapitel}
\begin{equation}
P(x)=\frac{1}{x^{s}\cdot\zeta(s)}
\end{equation}
Referenz: \ref{fig1:Beispieltext}
\bibliographystyle{plain}
\bibliography{KAM}
\end{document}
