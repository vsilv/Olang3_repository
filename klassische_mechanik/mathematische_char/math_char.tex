%Schriftgr{\"o}{\ss}e, Layout, Papierformat
\documentclass[12pt,a4paper]{article}

\usepackage[latin1]{inputenc}   %Umlaute erm{\"o}glichen
\usepackage[ngerman]{babel} %deutsches Paket

\usepackage{amssymb}%Mathemodus
\usepackage{amsmath}
\usepackage{latexsym}
\usepackage{algorithmic}
\usepackage{textcomp}
\usepackage[T1]{fontenc}
\usepackage{bm}% bold math
\usepackage{hyperref}
\usepackage{amsfonts}
\usepackage{graphics}
\usepackage{psfrag}
\usepackage[bf,small,sf]{caption}
\usepackage{graphicx}
\usepackage{epsfig}
\usepackage{rotating}
\usepackage[latin1]{inputenc}
\usepackage{color}
\usepackage[usenames,dvipsnames]{pstricks}
\usepackage{epsfig}
\usepackage{pst-grad} % For gradients
\usepackage{pst-plot} % For axes
%Seite einrichten
\usepackage[left=2cm,right=3cm,top=2cm,bottom=2cm,includeheadfoot]{geometry}
\parindent0mm % Unterdr{\"u}ckt die Absatzeinr{\"u}ckung
%Kopf- und Fu{\ss}zeilen
\usepackage{fancyhdr}                   %Paket f{\"u}r Kopf- und Fusszeile
\usepackage[section]{placeins}
\pagestyle{fancy} \fancyhf{}
%Kopfzeile
\fancyhead[L]{\small{\nouppercase{\leftmark}}}%
\fancyhead[R]{\small{Mathematische Charakterisierung von Chaos}
%Linie oben
\renewcommand{\headrulewidth}{0.5pt}
%Fu�zeile
\fancyfoot[C]{\small{\thepage}}

%%%%%%%%%%%%%%%%%%%%%%%%%%%%%%%%%%%%%%%%%%%%%%%%%%%%%%%%%
%ein paar Farben definieren
\definecolor{lightblue}{rgb}{0.3,0.85,1}
%Kommentarfunktion
\newcommand{\jnote}[1]{\tex_tcolor{red}{\tex_tit{[J: #1]}}}
\newcommand{\vnote}[1]{\tex_tcolor{blue}{\tex_tit{[V: #1]}}}
\newcommand{\fnote}[1]{\tex_tcolor{lightblue}{\tex_tit{[F: #1]}}}
\newcommand{\tnote}[1]{\tex_tcolor{green}{\tex_tit{[T: #1]}}}
%%%%%%%%%%%%%%%%%%%%%%%%%%%%%%%%%%%%%%%%%%%%%%%%%%%%%%%%%

%Dokumentbeginn
\begin{document}

%Titelseite
\begin{titlepage}
\begin{center}
\begin{large}
\begin{tabular}{ll}\\\\\\
\multicolumn{2}{p{14.0cm}}{\centering \huge\bf Mathematische Charakterisierung von Chaos
}\\[20mm]
\hline \\
\textbf{Name}:    & Annabelle Junger\\[6mm]
\textbf{Datum:}   & \today\\[5mm] %oder \today

\hline \\[35mm]

\end{tabular}

\end{large}
\end{center}
\end{titlepage}

\newpage
%Inhaltsverzeichnis
\tableofcontents
\newpage
\setcounter{page}{1}
\part{Wahloser Text}
\section{Definition eines dynamischen Systems}
Ein dynamisches System ist eine mathematische Beschreibung einer Zustandsver�nderung nach der Zeit,
formal ausgedr�ckt ein Tripel $(X, T, f)$ mit
\begin{itemize}
	\item $X =$ Zustandsraum; d.h. eine nichtleere Menge, genauer ein metrischer Raum mit Metrik $d$, 
		bestehend aus den Elementen, die sich mit der Zeit ver�ndern; 
z.B. Geldbetr�ge auf einem Sparbuch $ X = \mathbb{R}_{0}^{+}$
\item $T =$ Zeitraum; d.h. eine der Mengen, innerhalb derer sich jene Elemente von oben entweder nach
diskreten Zeitschritten ver�ndern oder ansonsten kontinuierlich; 
hier: jedes volle Jahr: $T = \mathbb{N}$
\item $f: T \times X \rightarrow X$,  $(t, x)\rightarrow ft(x) =$ stet. Funktion, nach der sich diese
 Elemente ver�ndern; hier: Multiplikation
mit 1,1 bei Zinssatz von $10\% \rightarrow f: \mathbb{N} \times \mathbb{R} \rightarrow \mathbb{R}_{0}^{+}$,
$ (n, r) \rightarrow  (1,1)^{n}r$ 
\item Dabei gilt:
	\begin{itemize}
		\item $ f(0, x) = f_0(x) = x$
		\item $ f(n, (f(m, x))) = f^n(f^m(x)) = f^{m+n}(x)$   f�r alle $x$ aus $X$ und $m$, $n$ aus $T$
	\end{itemize}
\end{itemize}
\section{Definition von Chaos}
Ein dynamisches System $(X, T, f)$ ist genau dann chaotisch, wenn folgende Bedingungen gelten
(wobei die ersten beiden die letzte, f�r das Chaos als essentiell betrachtete Eigenschaft implizieren):
\begin{itemize}
	\item Dichtheit der periodischen Punkte von $f$ in $X$; d.h., f�r alle $x$ aus $X$ existiert ein periodisches $y$  aus
$X$ beliebig nahe an $x$ (in Bezug auf die Metrik $d$)
\item Transitivit�t von $f$; d.h., f�r alle $x$, $y$ aus $X$ existiert ein $z$ beliebig nahe an $x$, dessen Orbit
beliebig nahe an dem von $y$ liegt
\item Sensibilit�t von $f$; d.h., f�r alle $x$ aus $X$ existiert ein $y$ aus $X$ beliebig nahe an x, so dass deren
Orbits an einer Stelle um einen Mindestwert voneinander differieren
\end{itemize}
\section{Definitionen versch. Orbits}
Der Orbit eines Elements aus der Zustandsmenge eines dynamischen Systems beschreibt dessen Ver�nderung nach
der Zeit. Er wird als Folge $(x_t)$ mit $x_t = f(t, x)$ verstanden, etwa $(x, f(x), f_2(x), f_3(x), ...)$.
Dabei unterscheidet man:
\begin{itemize}
	\item gemischt p-periodische Orbits, falls $p$, $t$ aus $T$ existieren, so dass $f_p(x_t) = x_t$ ; etwa
		$ (x_0, x_1, x_2, ..., x_t, x_{t+1}, {x_t}, x_{t+1}, ...)$ hier:$ p = 2$
$x_t$ nennt man in dem Falle einen periodischen Punkt
\item p-periodische Orbits, falls $t = 0$; etwa $ (x_0, x_1, x_0, x_1, ...)$ hier: $p = 2$
\item gemischte Fixpunktfolgen, falls $p = 1$; etwa $(x_0, x_1, x_2, x_3, x_3, x_3, ...)$ hier: $t = 3$
\\ $x_3$ nennt man in dem Falle einen Fixpunkt
\item Fixpunktfolgen, falls $t = 0$ und $p = 1$; etwa $(x_0, x_0, x_0, ...)$ 
\item Konvergente Orbits, falls dessen Folgeglieder gegen einen bestimmten Wert aus $X$ konvergieren;
 etwa $(256, 16, 4, 2, 1.414, 1.189, 1.091, 1.044, ...) \rightarrow  1$
\item Chaotische Orbits sonst
	\end{itemize}
\end{document}
